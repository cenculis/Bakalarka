\chapter{Záver}

Príkladmi z kapitoly \ref{Didaktické príklady} a \ref{PIDPID}, sme potvrdili funkčnosť AeroShieldu v API Arduino IDE, MATLAB a Simulink. AeroShield je teda použiteľný ako didaktická pomôcka a to aj napriek niektorým nedostatkom na softvérovom, ako aj hardvérovom rozhraní. 

Na AeroShielde je zaznamenávaný prúd, ktorý odoberá akčný člen sústavy. Toto meranie je pri malých zmenách prúdu pomerne nepresné a to z dôvodu ovládania motora PWM signálom. V budúcej verzii AeroShieldu môže byť použitý iný druh napájania motora, čo by malo za následok aj vylepšenie presnosti merania prúdu. Takto nameraný prúd by sa dal následne využívať na presnejšie riadenie výkonu motora, alebo na implementáciu PID regulácie na základe prúdu a nie uhlu kyvadla. 

Ďalším problémom pri PID regulácii AeroShieldu, je jeho zložité a dlhotrvajúce nastavenie parametrov. Pri príklade \ref{MATLABPID} na strane \pageref{MATLABPID}, bolo nastavenie parametrov najzložitejšie a strávil som pri ňom niekoľko desiatok hodín. Problémom bolo pomalšie vzorkovanie a teda aj malá zmena parametrov, spôsobila veľké zmeny na výstupe. 

Ďalším z problémov bolo priame prepojenie motora so Shieldom. V prípade nechceného pretočenia ramena kyvadla, sa napájacie káble zapletú na rameno a to spôsobý zamedzenie ďalšieho otáčania. Všetky didaktické príklady síce majú implementovanú softvérovú ochranu proti takémuto pretočeniu, avšak táto nie je 100\% účinná. 
Tento problém by vyriešila realizácia napájania pomocou konektora so zbernými krúžkami, avšak takýto konektor stojí v priemere 15\euro  a teda jeho aplikácia v nízko nákladovej učebnej pomôcke je otázna. Zároveň pomocou magnetu uloženého na konci kyvadla meriame jeho uhol. Použitý konektor by preto musel mať stredovú časť s možnosťou pripevnenia magnetu, alebo by sa uhol kyvadla musel merať iným spôsobom. 

Medzi vylepšenia nasledujúceho modelu AeroShieldu môžeme zaradiť úplnú zmenu podporného systému kyvadla. Uchytenie pomocou dvoch otočených \verb|V| konštrukcií, prepojených priečkou, by umožňovalo meranie natočenia na jednej strane priečky a druhou stranou by bolo realizované napájanie motora. Zaujímavým experimentom by bola možnosť hardvérovej zmeny smeru otáčania motora, pomocou prepínača. Pokiaľ by bola rýchlosť zmeny otáčania dostatočne rýchla (rádovo niekoľko desiatok milisekúnd), dal by sa realizovať príklad otočeného kyvadla, kedy je rameno kyvadla držané pomocou regulátora vo vzpriamenej polohe.   

Medzi dalšie vylepšenia pre budúcu prácu s AeroShieldom, môžeme zaradiť reguláciu pomocou iného algoritmu ako bol PID. Medzi ne môžeme zaradiť modelové prediktívne riadenie (MPC), Lineárno-kvadratické riadenie (LQ), Lineárne riadenie s premenlivým parametrom (LPV)... Pomocou týchto algoritmov by sme mohli dosiahnuť lepšie vlastnosti systému a tým pádom presnejšie riadenie. 

Všetky chyby a nedokonalosti dizajnu, ako aj neschopnost dokonalého nastavenia uhlu kyvadla pri PID regulácii,
neznemožňujú kvalitnú výuku s použitím AeroShieldu. Sú to skôr nápady a možnosti vylepšenia, ktoré môžu byť v budúcnosti na AeroShield implementované.