\noindent
\textbf{Názov práce:} AeroShield: Miniatúrny experimentálny modul aerokyvadla\\
\textbf{Kľúčové slová: } Arduino, AutomationShield, PID, AeroShield, Aero pendulum, softvér, hardvér, Arduino IDE, MATLAB, Simulink \\
\textbf{Abstrakt: } Cieľom bakalárskej práce je návrh experimentálneho modulu pre platformu Arduino. Tento modul má podobu externého shieldu, ktorý sa dá jednoducho pripojiť ku doskám Arduino a slúži na výučbu základov riadenia. Ich súčasťou je hardvérová a softvérová časť. V rámci bakalárskej práce boli navrhnuté dva moduly, AeroShield R2 a R3, ktoré replikujú experiment známy pod názvom aeropendulum. V rámci hardvérovej časti bolo zostavená doska plošných spojov, ako aj celkový model kyvadla. V softvérovej časti boli vytvorené didaktické príklady pre API Arduino IDE ako aj pre MATLAB a Simulink.\\

\noindent
\textbf{Title:}AeroShield: Miniature experimental module of aeropendulum \\
\textbf{Keywords: }  Arduino, AutomationShield, PID, AeroShield, Aero pendulum, software, hardware, Arduino IDE, MATLAB, Simulink\\
\textbf{Abstract: } The aim of this bachelor's thesis is to design an experimental module for the Arduino platform. This module takes the form of an external shield that can be easily connected to Arduino boards and is used to teach the basics of control engineering. Each module consists of hardware and a software part. As a part of this bachelor thesis, two modules were designed, the AeroShield R2 and R3. AeroShield was designed to replicate an experiment known as an aeropendulum. In the hardware part, the printed circuit board (PCB) was designed as well as the overall model of the pendulum. In the software part, didactic examples were created for the API Arduino IDE as well as for MATLAB and Simulink.
\cleardoublepage
