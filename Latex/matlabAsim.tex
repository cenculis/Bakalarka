
\subsection{MATLAB}
\label{MATLABPID}

Príklad na ukážku fungovania PID regulátora bol taktiež vytvorený v prostredí MATLAB a simulink. V prípade MATLABU využívame, na výpočet akčného zásahu, knižnicu PID.m, ktorá bola taktiež vytvorená v rámci programu AutomationShield. Na nastavenie parametrov PID slúži príkaz \code{PID.setParameters(Kp, Ti, Td, Ts)}. 

Na vzorkovanie využívame funkcie \verb|TIC a TOC |, ktoré merajú prejdený čas. Funkcia TIC zaznamenáva aktuálny čas a funkcia TOC používa zaznamenanú hodnotu na výpočet uplynulého času. Ak je splnená podmienka \code{if (toc>=Ts*k)}, povolí sa posun na nasledujúcu vzorku, pričom premenná \verb*|k|, udáva počet vykonaných vzoriek. Dáta sú postupne zapisované do poľa \code{ PIDresponse(k,:)=[r y u]}, a toto pole je vykresľované pomocou funkcie \code{plotLive(PIDresponse(k,:))}. Na konci sú všetky dáta uložené, a teda sú prístupné aj po ukončený programu. Kompletný zdrojový kód sa nachádza v prílohe \ref{AeroShieldPID.m}.

Regulácia PID v prostredí MATLAB potrebuje pre svoje fungovanie pomerne vysoký výpočtový výkon počítača, pretože výpočty prebiehajú na zariadení, s ktorým je Arduino prepojené. Na zariadení \cite{Notebook} v ktorom boli písané všetky didaktické príklady dosahujeme rýchlosť maximálne piatich vzoriek za sekundu tj. Ts=0.2s. Pri pomalšom vzorkovaní musíme spomaliť aj rýchlosť reakcie PID regulátora (napr. znížením proporcionálnej zložky), inak regulácia prestane byť možná. Výpočtová náročnosť sa dá znížiť, zamedzením vykresľovania grafu, alebo odstránením možnosti ukladania zaznamenaných dát. 

\begin{figure}[!tbh]
	\centering
	\includegraphics[width=90mm]{obr/jednotkovyskoskMAt.png}
	\caption{Reakcia systému na skokovú zmenu referenčnej hodnoty- MATLAB.}\label{OBRAZOK 2.6.1}
\end{figure}

\subsubsection{Výstupy}

Všetky výstupy z príkladu \ref{MATLABPID}, majú priamu funkciu vykresľovania grafov spolu s legendou výstupov. Tieto grafy majú taktiež definovaný rozsah zobrazovaných hodnôt na oboch osiach, ako aj pomenovania týchto osí. Na Obr. \ref{OBRAZOK 2.6.1} vidíme reakciu systému na jednotkový skok. Obrázok \ref{OBRAZOK 2.6.2} zobrazuje automatickú trajektóriu referenčnej hodnoty a Obr. \ref{OBRAZOK 2.6.3} trajektóriu manuálnu. 

Pri manuálnej trajektórii bola trikrát vnesená veľká chyba a to pomocou úderu do kyvadla. Ako je vidieť z grafu \ref{OBRAZOK 2.6.3}, ustálenie systému prebieha oveľa pomalšie ako v príklade \ref{Arduino IDE PID}. Oscilácia je pomerne vysoká a pretrváva po dobu cca 35 vzoriek, čo predstavuje približne 7 sekúnd. Táto skutočnosť je spôsobená pomalšou reakciou PID regulátora, na veľkú regulačnú odchýlku. Dlhší čas potrebný na ustálenie kyvadla je spôsobený predovšetkým pomalším vzorkovaním. 

\begin{figure}[!tbh]
	\centering
	\includegraphics[width=125mm]{obr/PIDautomaMat.png}
	\caption{Automatická trajektória.}\label{OBRAZOK 2.6.2}
\end{figure}
\begin{figure}[!tbh]
	\centering
	\includegraphics[width=125mm]{obr/pidmanualbuchh.png}
	\caption{Manuálna trajektória.}\label{OBRAZOK 2.6.3}
\end{figure}

\newpage 
\subsection{Simulink}

Vzhľadovo pôsobí príklad PID riadenia v API Simulink jednoducho a elegantne. Predpripravené bloky z knižnice AeroLibrary stačí v príklade pospájať podľa potreby a následne zvoliť vhodné parametre v maskách blokov. Prepájanie blokov slúži na spájanie vstupov s výstupmi, alebo na matematické operácie s premennými. Regulovanú sústavu v tomto príklade predstavuje blok \verb|AeroShield|, do ktorého vstupuje z bloku \verb|Reference read| percentuálna hodnota referenčnej trajektórie. Z bloku \verb|AeroShield| získavame ako výstup uhol kyvadla, ktorý využívame na výpočet regulačnej odchýlky.  

Blok \verb|Discrete PID Controller| predstavuje riadiaci systém PID regulátora. Hodnoty jednotlivých zložiek regulátora sú: P=0.011, I=200, D=8. Matematická reprezentácia výpočtového algoritmu ideálneho PID regulátora, je v tvare Rov. \ref{PIDSimulink}:

\begin{equation}\label{PIDSimulink}
	P\left(1+I*T_s\dfrac{1}{z-1}+D*\dfrac{1}{T_s}\dfrac{z-1}{z}\right)
\end{equation}


\begin{figure}[!tbh]
	\centering
	\includegraphics[width=125mm]{obr/SimulinkPID.png}
	\caption{Ukážka riadenia systému pomocou PID regulátora v API Simulink.}\label{OBRAZOK 2.6.10}
\end{figure}

\subsubsection{Výstupy}

\begin{figure}[!tbh]
	\centering
	\includegraphics[width=100mm]{obr/SimSkok.png}
	\caption{Reakcia systému na skokovú zmenu referenčnej hodnoty- Simulink.}\label{OBRAZOK 2.6.11}
\end{figure}

\begin{figure}[!tbh]
	\centering
	\includegraphics[width=125mm]{obr/SimulinkManualBuch.png}
	\caption{Manuálna trajektória.}\label{OBRAZOK 2.6.12}
\end{figure}