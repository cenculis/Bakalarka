\chapter{Didaktické príklady}

Pre AeroShield bolo v prostredí Arduino IDE, MATLAB a Simulink vytvorených niekoľko vzorových programov, ktoré demonštrujú všetky jeho funkcie a možnosti operácie. Programy sú rozdelené do dvoch veľkých skupín, konkrétne programy v otvorenej slučke bez spätnej väzby a programy v uzavretej slučke so spätnou väzbou. 

Ich rozdiel spočíva v tom že pri riadení bez spätnej väzby, hovoríme o ovládaní systému, kedy sa snažíme dosiahnuť žiadané hodnoty výstupov bez spätnej informácie o vykonaní procesu, alebo o jeho hodnote. V prípade riadenia so spätnou väzbou sa jedná o reguláciu. Pri regulácii sa kontroluje bezprostredný účinok riadenia, ktorý sa porovnáva so žiadanou hodnotou výstupu a na vyrovnanie ich vzájomnej chyby, sa okamžite vykonáva zásah do vstupných veličín. 

\section{Programy v otvorenej slučke, bez spätnej väzby}
\subsection{$AeroShield_OpenLoop.ino$}

Ako prvý príklad si ukážeme program s názvom \verb|AeroShield_OpenLoop.ino| napísaný v prostredí Arduino IDE. Hlavnou ideou tohoto programu je jednoduché ovládanie otáčok motorčeka kyvadla, pomocou potenciometra. Na začiatku programu inicializujeme hlavnú knižnicu AeroShieldu pomocou príkazu \verb|#include "AeroShield.h"|. Následne deklarujeme premenné, ktorých hodnoty budú vypisované na sériový monitor. 

\begin{lstlisting}[caption={AeroShield open loop dekleracia.},captionpos=b]
#include "AeroShield.h"       //  Inicializacia hlavnej kniznice

float startangle=0;           //  Premenna pre nulovy uhol
float lastangle=0;            //  Premenna pre maximalny uhol 
float pendulumAngle;          //  Uhol natocenia kyvadla
float referencePercent;       //  Hodnota potenciometra
float CurrentMean;	      //  Hodnota prudu odoberaneho motorom 
\end{lstlisting}

Nasleduje časť \verb|setup()|, v ktorej ako prvé, prebehne nastavenie rýchlosti sériovej komunikácie \verb|Serial.begin(115200)|. Číslo 115 200 predstavuje počet zmien, stavu z 0 na 1 resp. zo stavu high na stav low, za sekundu. Nasleduje funkcia \verb|AeroShield.begin()| ktorá sleduje prítomnosť magnetu, a pred nastaví potrebné premenné a funkcie pinov. Poslednou funkciou je kalibrácia kyvadla \verb|AeroShield.calibration()|, spolu s výpočtom začiatočného a koncového uhla. 

\begin{lstlisting}[caption={AeroShield open loop setup().},captionpos=b]
void setup() {                // Setup prebehne len jeden krat 
 Serial.begin(115200);       // Zaciatok seriovej komunikacie 
 AeroShield.begin(AeroShield.detectMagnet());  // Inicializacia AeroShieldu 
 startangle = AeroShield.calibration(AeroShield.getRawAngle());   // Kalibracia kyvadla
 lastangle=startangle+1024;  // Kalkulacia uhlu kyvadla pre map function
}
\end{lstlisting}

V časti \verb|loop()| sa program opakuje dookola. Ako prvé, prebehne mapovanie uhlu kyvadla pomocou funkcie \verb|AutomationShield.mapFloat()| a získaná hodnota uhlu sa vypíše na sériový monitor, spolu s názvom a premennou danej veličiny. Nasleduje čítanie hodnoty potenciometra, ktorá slúži na ovládanie akčného člena pomocou funkcie \verb|AeroShield.actuatorWrite()|. Na sériový port sa vypíše hodnota potenciometra, za ktorou nasleduje veľkosť prúdu odoberaného motorom \verb| AeroShield.currentMeasure()|. 

\begin{lstlisting}[caption={AeroShield open loop loop().},captionpos=b]
void loop() {
	pendulumAngle= AutomationShield.mapFloat(AeroShield.getRawAngle(),startangle,lastangle,0.00,90.00);    //  Mapovanie uhlu kyvadla 
	Serial.print("pendulum angle is: ");
	Serial.print(pendulumAngle);    
	Serial.print("\textdegree || ");
	
	referencePercent= AeroShield.referenceRead();  // Citanie potenciometra
	Serial.print("pot value is: ");
	Serial.print(referencePercent);  
	Serial.print("% || ");
	
	AeroShield.actuatorWrite(referencePercent); // Pohyb akcneho clenu
	
	CurrentMean= AeroShield.currentMeasure();  // Meranie prudu
	Serial.print("current value is: ");
	Serial.print(CurrentMean);   
	Serial.println("A || ");
}
\end{lstlisting}

\subsection{$AeroShield_Open_Loop.m$}


\section{Programy v uzatvorenej slučke, so spätnou väzbou}

sampling aj