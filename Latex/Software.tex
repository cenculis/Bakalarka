\section{Softvér}
\subsection{Arduino API}

Vývojové prostredie pre platformy arduino sa nazýva Arduino IDE\footnote[5]{Arduino Integrated Development Environment.} a využíva programovací jazyk C++ resp. jeho nadstavbu, s pridanými špecializovanými príkazmi a funkciami. C++ patrí medzi jeden z najviac používaných programovacích jazykov na svete. Je vhodný ako pre začiatočníkov, tak aj pre profesionálnych programátorov. V rámci projektu AutomationShield existuje niekoľko rôznych shildov, ktoré však často využívajú rovnaké funkcie a príkazy. Z tohoto dôvodu bola vytvorená knižnica $"$AutomationShield$"$, ktorá v sebe zahŕňa niekoľko ďalších, často využívaných knižníc a súborov. Jedná sa napríklad o funkcie potrebné pri riadení pomocou PID regulátora, vzorkovanie programu alebo mapovanie premenných. O jednotlivých funkciách v rámci knižnice $"$AutomationShield$"$ si povieme viacej pri opise didaktických príkladov v časti:\ref{Didaktické príklady}.

V rámci programovania knižníc využívame objektovo orientované programovanie \newline (OOP)\cite{oop}. OOP je výhodné z hľadiska prehľadnosti, úpravy funkcií, redukcie nepotrebného alebo duplicitného kódu a mnoho ďalšieho. Pri OOP vytvárame dátové štruktúry nazývané objekty. Objekty majú svoje vlastnosti, metódy a udalosti a pomocou kombinácii týchto vlastností vykonávajú určité naprogramované činnosti. 

Knižnice väčšinou rozdeľujeme do dvoch súborov. Prvým z nich je \verb|header| teda hlavička s koncovkou .h a druhý \verb|source| alebo zdrojový dokument s koncovkou .cpp. Header slúži ako akýsi navádzač a sklad pre premenné a funkcie, ktorý následne komunikuje so source dokumentom v ktorom sú uložené samotné funkcie.
Takéto rozdelenie súborov má za ciel zrýchlenie kompilácie programu. Knižnice vytvárame ako pre Arduino API tak aj pre vývojové prostredie MATLAB. 


\subsubsection{Header}

Header súbor má niekoľko náležitostí ktoré obsahuje. Na začiatok deklarujeme súbor samotný. Robíme to pomocou príkazu \verb|#define|. Avšak ak by sa takáto deklarácia nachádzala vo viacerých súboroch, a teda header súbor by sa načítal niekoľko krát, spôsobovalo by to problém pri kompilácii kódu. Z toho dôvodu používame funkciu \verb|Include guard|, ktorá zamedzuje niekoľkonásobnému načítaniu rovnakých súborov. 

Hneď za definovaním knižnice AeroShild.h môžeme vkladať ďalšie knižnice, ktoré sú potrebné pre funkcie danej knižnice, a to pomocou príkazu \verb|#include|. Môžeme si všimnúť že za príkazom \verb|#include| sa nachádzajú dva typy zátvoriek resp. znakov. Konvencia je taká že na preddefinované knižnice sa používajú hranaté zátvorky \verb|<nazovKniznice.h>| a na knižnice tvorené programátormi sa používajú úvodzovky \verb|"nazovDalsejKniznice.h"|.

Za knižnicami následne určujeme premenné, ktoré majú priradené fyzické čísla pinov na arduine. Tieto premenné potom využívame buď na posielanie, alebo na prijímanie signálov z daných pinov. Názvy týchto premenných sa snažíme voliť tak, aby bola na prvý pohľad jasná ich funkcia, alebo podľa všeobecne zaužívaných pravidiel. V teórii riadenia sa na označenie vstupov používa písmeno \verb|R| a na označenie výstupov \verb|U,Y|. Príkaz \verb|#endif| vkladáme až na úplný záver header súboru.

\begin{lstlisting}[caption={Ukážka zdrojového kódu headeru.},captionpos=b]
#ifndef AEROSHIELD_H	 	 // Pokial nie je definovana AEROSHIELD_H
#define AEROSHIELD_H	 	 // Definuj kniznicu AEROSHIELD_H

#include "AutomationShield.h"    // Hlavna kniznica AutomationShieldu
#include <Wire.h>                // Kniznica potrebna pre komunikaciu I2C
#include <Arduino.h>		 // Zakladna arduino kniznica

#define AERO_RPIN A3             // Vstup z potenciometra
#define VOLTAGE_SENSOR_PIN A2    // Vstup pre meranie prudu 
#define AERO_UPIN 5              // Aktuator

----------------------Zdrojovy kod----------------------

#endif			   	 // Koniec if podmienky 
\end{lstlisting}



V časti \verb|Zdrojovy kod|, vytvárame \verb|class| v preklade triedu, ktorá v sebe zahŕňa funkcie a premenné, ktoré sa nazývajú \verb|objects|, teda objekty. Class obsahuje podmnožinu objektov, ktoré vieme prepájať a spájať vo väčšie celky, vďaka čomu vieme dosiahnuť veľmi komplexné funkcie. Týmto funkciám a premenným vieme obmedziť prístup resp. ich dosah v rámci programu, pomocou modifikátorov prístupu (access modifiers). Tieto modifikátory delíme do štyroch skupín. Základný modifikátor je \verb|default|, teda akýsi predvolený prístup, ktorý nadobúdajú všetky funkcie a premenné automaticky, pokiaľ nezvolíme iný modifikátor. Ďalšími modifikátormi sú \verb|public|, teda funkcie a premenné verejné, prístupné v triede aj mimo nej a \verb|privat|, teda súkromné, prístupné len v danej triede. Posledným modifikátorom prístupu je \verb|protected| teda prístup chránený. V header súbore sa môže nachádzať jedna, alebo viacero tried, záleží to od logicky deliteľných úsekov kódu, alebo od preferencie programátora. Deklarácia triedy s objektami vyzerá nasledovne: 

\begin{lstlisting}[caption={Triedy a objekty.},captionpos=b]
	class AeroShield{		// Deklaracia triedy
		public :		// Verejna cast
		void FirstObject();	// Deklaracia funkcie
		
		private :		// Sukromna cast
		float FirstVariable;	// Deklaracia premennej
	};				// Koniec triedy
\end{lstlisting}

V tomto prípade sa trieda nazýva AeroShield a má v sebe jednu funkciu s názvom \verb|FirstObject()| v časti public a jednu premennú \verb|FirstVariable|, typu float, v časti pivate. Rozdelenie na public a privat má zmysel hlavne v prípade ak chceme mať zadefinované isté premenné, pri ktorých nechceme aby sa dala externe zmeniť ich hodnota alebo typ. V prípade privat, takáto zmena nie je možná, jediná možnosť ako premennú zmeniť, je jej ručné prepísanie v súbore. V časti private deklarujeme funkcie ktoré následne využívame v rámci triedy a slúžia ako pomocné funkcie pri tvorbe komplexnejších častí kódu. V časti public sú funkcie viditeľné a schopné interagovať s inými triedami ako aj s inými knižnicami. 

\newpage
\subsubsection{Source}

Keďže všetky potrebné knižnice sa už definovali a načítavali v súbore header, stačí nám už len časti source a header prepojiť. Urobíme tak pomocou príkazu \code{#include "AeroShield.h"}, ktorý vložíme na začiatok súboru. Ďalej v súbore deklarujeme jednotlivé funkcie, ktoré zapisujeme pomocou už spomínaného classu, dátového typu a názvu funkcie v podobe:

\begin{lstlisting}[caption={Source volanie funkcie.},captionpos=b]
typFunkcie AeroShield::nazovFunkcie()
\end{lstlisting}

V tomto prípade je AeroShield názov classu, nazovFunkcie hovorí sám za seba. Dátové typy funkcií poznáme rôzne. Vyberáme si ich na základe potreby ako chceme aby funkcia reagovala resp. aké hodnoty by mala prenášať. Dátové typy poznáme nasledovné\cite{datovetypy} (všetky hodnoty sú platné pre arduino UNO, pre iné typy arduina sa hodnoty môžu líšiť): 

\begin{table}[!ht]
	\centering
	\begin{tabular}{|p{0.22\textwidth} | p{0.22\textwidth} |p{0.22\textwidth} |p{0.22\textwidth} |}
		\hline
		\thead{dátový typ} & \thead{vlastnosti} & \thead{dátový typ} & \thead{vlastnosti} \\ \hline
		\textbf{array} & skupina premenných s priradeným indexom. Maximálna veľkosť je obmedzená veľkosťou pamäte RAM & \textbf{short }& 16 bitové celé čísla \\ \hline
		\textbf{boolean} & má buď hodnotu 0-nepravda, alebo 1-pravda & \textbf{char array }& spojenie viacerých dát typu char ukončené hodnotou null \\\hline
		\textbf{byte} & 8 bitové čísla od 0 do 255 & \textbf{string-object} & podobná funkcia ako object v header súbore \\ \hline
		\textbf{double} & rovnaké ako float & \textbf{unsigned char} & 8-bit znaky od 0 do 255 \\ \hline
		\textbf{float} & 32 bitové desatinne čísla $\pm$3.4028235E$+38$ & \textbf{unsigned int }& 16 bitové kladné celé čísla od 0 do 2$^{16}$-1 \\\hline
		\textbf{char} & 8 bit ascii tabulka & \textbf{unsigned long} & 32 bitové kladné celé čísla od 0 do 2$^{32}$-1 \\\hline
		\textbf{int} & 16 bitové celé čísla & \textbf{void }& nevracia naspäť žiadne informácie \\ \hline
		\textbf{long }& 32 bitové celé čísla & \textbf{word} & 16-bit číslo bez znamienka \\ \hline
    \end{tabular}
\caption{Dátové typy}
\label{tabulka typov}
\end{table}

\subsubsection{Popis použitých funkcií z knižnice AutomationShield}

Ako už bolo spomenuté, knižnica AutomationShield ponúka najviac používané funkcie, ktoré sa využívajú takmer na každom shielde. Pri rôznych veľkostiach a rozsahoch číselných stupníc, je dobré vyjadrovať hodnoty v percentách, namiesto ich absolútnej hodnoty. Arduino ponúka funkciu \verb|map()|, ktorá však pracuje len s dátovým typom integer. Aby sme docielili vyššiu presnosť, potrebujeme mapovať dátový typ float. Na tento účel nám slúži funkcia mapFloat do ktorej vstupuje veličina x, ktorej priradíme požadované hodnoty. Funkcie funguje na základe princípu lineárneho mapovania\cite{linearMap}.  

\begin{lstlisting}[caption={Zdrojový kód funkcie mapFloat.},captionpos=b]
float AutomationShieldClass::mapFloat(float x, float in_min, float in_max, float out_min, float out_max) 
{
return (x - in_min) * (out_max - out_min) / (in_max - in_min) + out_min; 
}
\end{lstlisting}

Ďalšou z funkcií použitých z knižnice AutomationShield je serialPrint. Funkcia vypisuje zvolený text na sériový monitor arduina. 

\begin{lstlisting}[caption={Zdrojový kód funkcie serialPrint.},captionpos=b]
 void AutomationShieldClass::serialPrint(const char *str){   
	#if ECHO_TO_SERIAL           // Pokial je tato funkcia povolena                       
	Serial.print(str);           // Vypis na seriovy monitor                       
	#endif 		// Koniec
}
\end{lstlisting}

\subsubsection{Popis použitých funkcií z knižnice AeroShield}


Keďže na AeroShielde využívame senzor hall efektu, musíme s ním v prvom rade nadviazať komunikáciu pomocou sériovej komunikácie I$^{2}$C. Protokol I$^{2}$C využíva na odosielanie a prijímanie údajov dva vodiče resp. dve linky: 
\begin{itemize}
\item sériovú dátovú linku (SDA-serial data), cez ktorú sa posielajú údaje, 
\item sériovú hodinovú linku (SCL-serial clock), na ktorú arduino v pravidelných intervaloch posiela impulzy. 
\end{itemize}

Hodinový pin udáva tempo komunikácie a je ovládaný mastrom. Mení stav v pravidelných impulzoch z low-nízkeho na high-vysoký stav. Pri každej takejto zmene je na dátový pin poslaný jeden bit informácie. Tieto bity najskôr obsahujú adresu zariadenia slave s ktorým chce master komunikovať, následne sa odosielajú bity príkazov. Keď sa táto informácia celá odošle, slave vykoná požiadavku a ak je to vyžadované, môže spätne mastrovi poslať údaje. Všetky tieto bity informácií sa posielajú na linke SDA\cite{idvac}.

I$^{2}$C funguje na princípe master-slave, kedy master je nadriadený a slave je podriadené zariadenie, s ktorým master komunikuje. Master môže naraz komunikovať s viacerými zariadeniami a to na základe jedinečných adries zariadení, ktoré sa medzi sebou striedajú v komunikácii.   

\subsubsection{Funkcia readOneByte()}

Pre naše účely postačuje vedieť čítať jeden alebo dva bajty informácií. Z toho dôvodu sme vytvorili dve funkcie: \code{int AeroShield::readOneByte()} a \newline\code{word AeroShield::readTwoBytes()}. Funkcia \code{int AeroShield::readOneByte()}, ako jej názov napovedá, získava 1 bajt informácii zo senzoru. Túto funkciu využívame napríklad na čítanie polohy kyvadla. 

\begin{lstlisting}[caption={Zdrojový kód funkcie readOneByte.},captionpos=b]
int AeroShield::readOneByte(int in_adr)         
{
	int retVal = -1;	 // Zadefinovanie pomocnej premennej
	Wire.beginTransmission(_ams5600_Address);// Zaciatok komunikacie 
	Wire.write(in_adr);	// Poziadavka na zaznamenianie uhlu kyvadla 
	Wire.endTransmission();	// Koniec komunikacie zo strany mastra
	Wire.requestFrom(_ams5600_Address, 1);	// Ziadost na odpoved  
	while (Wire.available() == 0);	// Cakaj pokial odpoved nepride  
	
	retVal = Wire.read();	// Zaznamenanie odpovede 
	
	return retVal;	// Zaslanie odpovede 
}
\end{lstlisting}

Ako môžeme vidieť v kóde funkcie, master najskôr osloví zariadenie slave, pomocou jeho adresy. Pri tomto konkrétnom čipe je adresa zariadenia v hexadecimálnej podobe 0x36. Následne zašle od výrobcu predprogramovanú žiadosť, ktorá zaznamená aktuálnu polohu magnetu resp. v našom prípade kyvadla. Následne je komunikácia ukončená a je zaslaná požiadavka na odpoveď zo strany slave zariadenia. Táto odpoveď je zaznamenaná a odoslaná späť, na miesto z ktorého bola funkcia privolaná.

\subsubsection{Funkcia readTwoBytes()}

Funkcia \code{word AeroShield::readTwoBytes()} je podobná predošlej funkcii, s rozdielom, že získané su dva bajty informácií narozdiel od jedného a na konci funkcie prebieha posun bitov\footnote[7]{Bitový posun je operácia vykonávaná so všetkými bitmi binárnej hodnoty, pri ktorej sa posúvajú o určený počet miest doľava alebo doprava\cite{biteShift}.}. 

\begin{lstlisting}[caption={Zdrojový kód funkcie readTwoBytes.},captionpos=b]
word AeroShield::readTwoBytes(int in_adr_hi, int in_adr_lo)        
{
	word retVal = -1;		// Zadefinovanie pomocnej premennej
	/* citanie "Low" bajtu */
	Wire.beginTransmission(_ams5600_Address);// Zaciatok komunikacie 
	Wire.write(in_adr);	// Poziadavka na zaznamenianie uhlu kyvadla 
	Wire.endTransmission();	// Koniec komunikacie zo strany mastra
	Wire.requestFrom(_ams5600_Address, 1);	// Ziadost na odpoved  
	while (Wire.available() == 0);	// Cakaj pokial odpoved nepride  

	int low = Wire.read();     	// Ulozenie prveho bajtu 
	/* citanie "High" bajtu */
	Wire.beginTransmission(_ams5600_Address);// Zaciatok komunikacie 
	Wire.write(in_adr);	// Poziadavka na zaznamenianie uhlu kyvadla 
	Wire.endTransmission();	// Koniec komunikacie zo strany mastra
	Wire.requestFrom(_ams5600_Address, 1);	// Ziadost na odpoved  
	while (Wire.available() == 0);	// Cakaj pokial odpoved nepride  
	
	word high = Wire.read();   	// Ulozenie druheho bajtu 
	
	high = high << 8;          	// Posun bitov
	retVal = high | low;
	
	return retVal;	   	  	// Zaslanie odpovede 
}
\end{lstlisting}

\subsubsection{Funkcia detectMagnet()}

Ďalšou dôležitou funkciou, je zistiť prítomnosť magnetu na kyvadle. Túto úlohu vykonáva funkcia \code{int AeroShield::detectMagnet()}. Využívame ju vždy pri inicializácii AeroShieldu, na to aby sme zistili či nenastali problémy s magnetom, alebo so senzorom samotným. Funkcia komunikuje so senzorom, a na základe výstupu vieme určiť či bol magnet detegovaný. Funkcia nám vráti na výstupe 1- pokiaľ sa magnet nachádza pri senzore a 0- pokiaľ magnet nie je prítomný, alebo je príliš vzdialený od senzoru. 

\begin{lstlisting}[caption={Zdrojový kód funkcie detectMagnet.},captionpos=b]
int AeroShield::detectMagnet() 
{
	int magStatus;		// Pomocna premenna  
	int retVal = 0;		// Pomocna premenna
	magStatus = readOneByte(_stat);	// Prebieha komunikacia so senzorom                        
	
	if (magStatus & 0x20)		// Pokial je podmeinka splnena vrat 1, pokial nie je splnena vrat 0 
	retVal = 1;
	
	return retVal;			// Zaslanie odpovede 
}
\end{lstlisting}

\subsubsection{Funkcia getMagnetStrength()}

Pre správnosť fungovania hall senzoru je dôležité dodržať správnu vzdialenosť magnetu od senzoru. Výrobca udáva že najideálnejšia vzdialenosť je 0.5-3mm, v závislosti na sile a veľkosti magnetu. Bolo by nepraktické túto vzdialenosť merať ručne, použijeme preto funkciu \code{int AeroShield::getMagnetStrength()}. Môžeme si všimnúť že táto funkcia používa rovnaký príkaz na komunikáciu so senzorom, ako aj funkcia \code{detectMagnet()} a to síce \verb|_stat|. Z toho vyplýva že \code{detectMagnet()} kontroluje nielen prítomnosť magnetu, ale aj jeho správnu vzdialenosť. Pokiaľ teda dostaneme z funkcie \code{detectMagnet()} ako výsledok 1, vieme že magnet je prítomný a zároveň v ideálnej vzdialenosti. Funkcia \code{getMagnetStrength()} je teda iba doplňujúcou funkciou, ktorá nám určí či je magnet moc blízko senzoru, výsledná hodnota výstupu bude 3, alebo naopak, je magnet moc vzdialený a výstupná hodnota bude 1. 

\begin{lstlisting}[caption={Zdrojový kód funkcie getMagnetStrength.},captionpos=b]
int AeroShield::getMagnetStrength()   
{
	int magStatus;                  // Pomocna premenna 
	int retVal = 0;                 // Pomocna premenna
	magStatus = readOneByte(_stat);	// Prebieha komunikacia so senzorom       
	
	if (detectMagnet() == 1)	// Pokial je splnena podmienka detectMagnet()
	{
		retVal = 2;  // Vrat 2, magnet je v idelnej vzdialenosti
		if (magStatus & 0x10)
		retVal = 1;  // Vrat 1, magnet je v prilis daleko
		else if (magStatus & 0x08)
		retVal = 3;  // Vrat 3, magnet je v prilis blizko
	}
	
	return retVal;                  // Zaslanie odpovede  
}
\end{lstlisting}

\subsubsection{Funkcia getRawAngle()}

Poslednou funkciu komunikácie s hall senzorom je \code{word AeroShield::getRawAngle()}. Funkcia slúži na čítanie samotného uhlu kyvadla. Výsledkom tejto funkcie je číslo s rozsahom 12bitov, teda číslo od 0 do 4096 ktoré udáva momentálnu polohu kyvadla. O tomto senzore, ako aj o jeho fungovaní sme už hovorili bližšie v časti \ref{meruhl}. 

\begin{lstlisting}[caption={Zdrojový kód funkcie getRawAngle.},captionpos=b]
word AeroShield::getRawAngle() 
{
	return readTwoBytes(_raw_ang_hi, _raw_ang_lo); // Prebieha komunikacia so senzorom, ktory rovno vrati vysledok pomocou prikazu return 
}
\end{lstlisting}

\subsubsection{Funkcia begin()}

Prvou z funkcií mimo komunikácie s hall senzorom je \code{float AeroShield::begin(bool isDetected)}, do ktorej vstupuje výsledok z funkcie \code{detectMagnet()} ako premenná \verb|isDetected|. Funkcia \code{begin()} nastaví pin potrebný na ovládanie akčného člena, pomocou príkazu \verb|pinMode|, ako výstup, teda \verb|OUTPUT|. Zároveň inicializuje sériovú komunikáciu I$^{2}$C. Príkaz na započatie komunikácie I$^{2}$C sa pri rôznych typoch dosiek, resp. architektúr mikroradiča Arduino líši. Použijeme preto podmienku \verb|#ifdef| za ktorou nasleduje typ architektúry daného mikroradiča a príslušný príkaz pre začiatok sériovej komunikácie I$^{2}$C. V prípade Arduino UNO, je to príkaz \verb|Wire.begin()|. 

Zároveň je vo funkcii \code{begin()}, pomocou if podmienky, kontrolovaná premenná \verb|isDetected|. Pokiaľ bol magnet detegovaný, vypíše na sériový port správu "Magnet detected$"$ a while\footnote[8]{Cyklus while sa bude opakovať nepretržite, pokiaľ sa výraz vnútri zátvoriek () nestane nepravdivým.} cyklus, sa pomocou príkazu \verb|break| ukončí. Pokiaľ magnet detegovaný nebol, vypíše "Can not detect magnet", no while cyklus pokračuje.  


\begin{lstlisting}[caption={Zdrojový kód funkcie begin.},captionpos=b]
float AeroClass::begin(void){                       
	bool isDetected = AeroShield.detectMagnet(); 
	// Detekcia magnetu 
	pinMode(AERO_UPIN,OUTPUT);	// Pin aktuatora
	
	#ifdef ARDUINO_ARCH_AVR		// Pre dosky s architekturov AVR
	Wire.begin();			// Pouzi objekt Wire
	#elif ARDUINO_ARCH_SAM      	// Pre dosky s architekturov SAM
	Wire1.begin();			// Pouzi objekt Wire1
	#elif ARDUINO_ARCH_SAMD     	// Pre dosky s architekturov SAMD
	Wire.begin();			// Pouzi objekt Wire
	#endif
	
	if(isDetected == 0 ){       	// Pokial magnet nie je detegovany
		while(1){               // While podmienka
		 if(isDetected == 1 ){	// Pokial sa magnet deteguje
		  AutomationShield.serialPrint("Magnet detected \n");	
		break;		    // Koniec while podmienky
		}
		else{               // Pokial magnet nie je detegovany
		  AutomationShield.serialPrint("Can not detect magnet \n");
			}
		}
	}       
} 
\end{lstlisting}

\subsubsection{Funkcia calibration()}

Ďalšou v poradí je funkcia \code{calibration()}. Slúži na prepočet a zaznamenanie nulovej polohy kyvadla, teda takej polohy v ktorej sa kyvadlo nachádza, pokiaľ motorček nie je poháňaný. V ideálnom prípade by sa kyvadlo vždy po vypnutí motora vrátilo do rovnakej východzej polohy. Avšak kyvadlo je prepojené s motorom a shieldom pomocou káblov, ktoré tvoria odpor a teda kyvadlo sa vždy zastaví v inej nulovej polohe. Funkcia \code{calibration()} teda slúži na zaznamenanie tejto nulovej polohy a všetky následné výpočty sa odvolávajú práve na túto hodnotu. 

Do funkcie vstupuje hodnota aktuálneho uhlu kyvadla, z funkcie \code{getRawAngle()} ako premenná \verb|RawAngle|. Funkcia na začiatok vypíše text "Calibration running..." a motorček zapneme na štvrť sekundy na výkon 20\%. Kyvadlo sa začne kývať a počkáme štyri sekundy, pokiaľ sa ustáli. Keď je kyvadlo ustálené zaznamenáme jeho hodnotu do premennej \verb|startangle|. Následne prebieha for cyklus ktorý zvukovo informuje o dokončenej kalibrácii pomocou troch pípnutí. Využívame tu jav, pri ktorom motorček vydáva vysoký tón, pokiaľ je do neho privádzaný nízky PWM signál. 

\begin{lstlisting}[caption={Zdrojový kód funkcie calibration.},captionpos=b]
float AeroShield::calibration(word RawAngle) {  
	AutomationShield.serialPrint("Calibration running...\n");  
	startangle=0;              // Vynulovanie premennej 
	analogWrite(AERO_UPIN,50); // Spustenie motora na vykon 20%
	delay(250);                // Cakaj 0.25s 
	analogWrite(AERO_UPIN,0);  // Vypnutie motora
	delay(4000);               // Cakaj 4s
	
	startangle = RawAngle;     // Uloz hodnotu nuloveho uhla
	analogWrite(AERO_UPIN,0);  // Poistne vypnutie motora 
	for(int i=0;i<3;i++){      // Funkcia na zvukovu signalizaciu
		analogWrite(AERO_UPIN,1);     // Zapnutie motora
		delay(200);                   // Cakaj
		analogWrite(AERO_UPIN,0);     // Vypnutie motora
		delay(200);                   // Cakaj
	}
	
	AutomationShield.serialPrint("Calibration done");
	return startangle;                    // Vrat hodnotu 
}
\end{lstlisting}

\subsubsection{Funkcia convertRawAngleToDegrees()}

Ako sme už spomínali v sekcii \ref{meruhl}, zaznamenávaná hodnota uhlu kyvadla je v rozmedzí od 0 do 4096 a tieto hodnoty priamo korešpondujú zo stupňami od 0\textdegree  do 360\textdegree. 1\textdegree  predstavuje hodnotu približne 11.77 vo formáte raw. Funkcia \code{convertRawAngleToDegrees()} teda len zoberie raw hodnotu uhlu a vynásobí ju hodnotou $\dfrac{360\textdegree}{4096}$ =0.087\textdegree. Funkcia teda naspäť vráti prepočítanú hodnotu uhla kyvadla v stupňoch. 

\begin{lstlisting}[caption={Zdrojový kód funkcie convertRawAngleToDegrees.},captionpos=b]
float AeroShield::convertRawAngleToDegrees(word newAngle) {  
	float ang = newAngle * 0.087;    // 360/4096=0.087 x rawHodnota                             
	return ang;                      // Vrat hodnotu
}
\end{lstlisting}


\subsubsection{Funkcia referenceRead()}

Funkcia \code{referenceRead()} slúži na čítanie hodnoty z potenciometra, ktorý sa nachádza na shielde, a jeho následne premapovanie do percentuálnej podoby. Potenciometer využívame na manuálne ovládanie AeroShieldu a v ďalších funkciách, sa využíva hlavne jeho percentuálna hodnota. Vrátená hodnota je typu float, v škále od 0.0\% do 100.0\%. 

\begin{lstlisting}[caption={Zdrojový kód funkcie referenceRead.},captionpos=b]
	  float AeroShield::referenceRead(void) {      
		referencePercent = AutomationShield.mapFloat(analogRead(AERO_RPIN), 0.0, 1024.0, 0.0, 100.0);  
		 // Premapovanie originalnej hodnoty 0.0-1023 na percentualny rozsah 0.0-100.0
		return referencePercent;      // Vrat percentualnu hodnotu 
	}
\end{lstlisting}
	
\subsubsection{Funkcia actuatorWrite()}	
	
Na ovládanie motora resp. jeho otáčok, používame funkciu \code{actuatorWrite()}. Do funkcie vstupuje percentuálna hodnota výkonu motora, táto hodnota je premapovaná na PWM signál, potrebný na správne ovládanie motora. Tento signál následne vstupuje do ochrannej funkcie \code{constrainFloat()}, ktorá zabezpečí aby sa hodnota PWM signálu mohla pohybovať len v rozmedzí od 0.0 do 255.0. Táto hodnota je potom zapísaná na príslušný pin motora, v našom prípade je to \verb|AERO_UPIN|. 
	
	
\begin{lstlisting}[caption={Zdrojový kód funkcie actuatorWrite.},captionpos=b]
	void AeroShield::actuatorWrite(float PotPercent) {   
		float mappedValue = AutomationShield.mapFloat(PotPercent, 0.0, 100.0, 0.0, 255.0);       
		// Vstupna percentualna hodnota 0.0-100.0 premapovana na hodnoty 0.0-255.0
		mappedValue = AutomationShield.constrainFloat(mappedValue, 0.0, 255.0);  
		// Bezpecnostna funkcia obmedzenia premapovanej hodnoty
		analogWrite(AERO_UPIN, (int)mappedValue);  // Zapisanie hodnoty na pin
	}
\end{lstlisting}
	
	
\subsubsection{Funkcia currentMeasure()}	
	
Poslednou funkciou AeroShieldu je \code{currentMeasure()}. Funkcia slúži na zaznamenávanie prúdu, ktorý odoberá motor kyvadla. Fungovanie snímača je opísané v sekcii \ref{merprud}. Funkcia slúži na prepočítanie zaznamenanej hodnoty napätia na prúd. Funkcia beží vo for cykle, ktorý sa opakuje \verb|repeatTimes|-krát, z dôvodu presnejšieho merania, keďže meraná hodnota sa priveľa mení, čo skresľuje výslednú hodnotu. Vďaka for cyklu získame priemernú hodnotu prúdu. 

Výsledná hodnota prídu prechádza úpravami, pomocou dvoch korekčných premenných, ktoré sme získali vďaka meraniam prúdu pomocou multimetra, a následným porovnaním hodnôt zo senzora. Porovnaním hodnôt sme získali veľkosti korekčných premenných, vďaka čomu sa naše merania zhodujú s meraniami pomocou multimetra, na dve desatinné miesta. 

Nepresnosť vzniká v dôsledku ovládania motora PWM signálom. Prúd je teda prerušovaný a jeho meraná hodnota v čase skáče medzi hodnotou skutočnou a nulovou. Musíme preto vykonávať viacero meraní, ktoré sú sčítavaní. Z tejto sčítanej hodnoty nameraných prúdov získame priemernú hodnotu prúdu, na ktorú sú následne aplikované korekčné premenné.  

Prevod z napätia na prúd robíme podľa pokynov uvedených v zdrojovom dokumente senzoru. Pri tomto prevode využívame hodnotu shunt rezistora \verb|RS|, ako aj hodnotu rezistora \verb|RL|, ktorý priamo premieňa prúd na napätie. 

Kedže prúd nemôže vykazovať záporné hodnoty(iba pokiaľ prúdi opačným smerom, ako sme zamýšľali), na záver funkcie ešte prechádza if podmienkou, ktorá zmení záporné hodnoty prúdu na hodnotu 0.0A.  
	
\begin{lstlisting}[caption={Zdrojový kód funkcie currentMeasure.},captionpos=b]	
float AeroShield::currentMeasure(void){  
	for(int i=0 ; i<repeatTimes ; i++){     // For cyklus
	voltageValue= analogRead(VOLTAGE_SENSOR_PIN);     
	// Citanie hodnoty zo senzoru INA169 
	voltageValue= (voltageValue * voltageReference) / 1024;    
	// Mapovanie hodnoty zo senzoru, na realnu hodnotu napatia(referencne napatie je 5V)
	current= current + correction1-(voltageValue / (10 * ShuntRes));    
                // Vzorec na vypocet prudu
                // Is = (Vout x 1k) / (RS x RL)
}                                                                         	
	float currentMean= current/repeatTimes;   
	// Vypocet priemernej hodnoty  
	currentMean= currentMean-correction2;       // Korekcia
	if(currentMean < 0.000) currentMean= 0.0;                 
	                    // Korekcia nulovej hodnoty
	current= 0;         // Vynulovanie pomocnych premennych   
	voltageValue= 0;    // Vynulovanie pomocnych premennych   
	return currentMean; // Vrat hodnotu prudu v amperoch
}
\end{lstlisting}
	
\subsection{MATLAB}	
	
Programovanie príkladov využitia AeroShieldu prebiehalo aj v prostredí programu MATLAB. V tomto programe, rovnako ako pri Arduino IDE, vieme príkazy a funkcie zapisovať do knižnice, odkiaľ si potom jednotlivé položky voláme do hlavného kódu. Z toho dôvodu bola zostavená knižnica \verb|AeroShield.m|. Knižnice v MATLABE môžu mať podobu súborov \verb|Header| a \verb|Source|, alebo funkcie zapíšeme len do jedného súboru s koncovkou \verb|.m|. 

V našom prípade sa jedná o jeden súbor na ktorého začiatku definujeme triedu súboru, pomocou príkazu \code{classdef AeroShield < handle ... end}. Za príkazom \verb|classdef| nasleduje názov našej triedy \verb|AeroShield| a porovnávací symbol $<$, za ktorým nasleduje $"$super trieda$"$ \verb|handle|. Super, alebo nad-trieda \verb|handle| je abstraktná trieda, ktorej inštancia sa nedá vytvoriť priamo. Táto trieda sa využíva na odvodenie iných tried, ktoré sú už konkrétnymi triedami, ktorých inštancie sú objektmi handle. 

Každá trieda môže následne obsahovať jeden alebo viacero triednych blokov: 
\begin{itemize}
	\item Properties(atribúty) ...end- Vymedzuje blok kódu, ktorý definuje vlastnosti s rovnakými nastaveniami atribútov.
	\item Methods(atribúty)    ...end- Má rovnaký názov ako trieda v ktorej sa nachádza a vracia inicializovaný objekt triedy.
	\item Events(atribúty)     ...end- Vymedzuje blok kódu, v ktorom nastane istá preddefinovaná udalosť, napr. zmena stavu. 
	\item Enumeration(atribúty)...end- Vytvorí zoznam hodnôt/premenných. 
\end{itemize}

Máme teda definované dva triedne bloky typu properties. V prom bloku sa nachádzajú objekty pre dosku arduino a hall senzor. V druhom bloku definujeme všetky premenné ktoré budeme ďalej v kóde využívať. Jedná sa hlavne o názvy a čísla pinov, ktoré používame na čítanie alebo zapisovanie hodnôt, ako aj o premenné, na výpočet prúdu odoberaného motorom. 

\begin{lstlisting}[caption={Knižnica AeroShield.m properties.},captionpos=b]
classdef AeroShield < handle

properties
	arduino;		% objekt dosky arduino 
	as5600;			% objekt hall senzoru
end

properties(Constant)
	AERO_UPIN = 'D5';          % pin aktuatora 
	AERO_RPIN = 'A3';          % pin potenciometra
	VOLTAGE_SENSOR_PIN = 'A2'; % pin na meranie prudu
	
	voltageReference = 5.0;    % referencne hodnota napatia 
	ShuntRes = 0.1;            % hodnota shunt rezitora v ohmoch
	correction1 = 4.1220;	   % korekcia  				  
	correction2 = 0.33;        % korekcia
	repeatTimes = 50;          % pocet cyklov pre vypocet priemer hodnoty prudu 
end		
\end{lstlisting}
		
Nasleduje triedy blok \verb|methods|, v ktorom sú všetky funkcie, ktoré budeme volať do hlavnej časti kódu. V bloku \verb|methods| sa nachádzajú jednotivé funkcie, ktoré definujeme ako \code{function nazovFunkcie(AeroShieldObject)} a za týmto zápisom, píšeme samotné príkazy. Keďže logika funkcií je rovnaká ako pri súbore \verb|Source|, nebudeme si funkcie vysvetľovať po jednom. Dôležité je poznamenať, že pokiaľ nám do funkcie nejaká premenná vstupuje, zapisujeme ju v zátvorke za príkazom funkcie \code{function nazovFunkcie(AeroShieldObject, vstupnaPremenna)}. Naopak, pokiaľ z funkcie nejaká premenná vychádza, túto premennú deklarujeme pred názvom funkcie \code{function vystupnaPremenna = nazovFunkcie(AeroShieldObject)}. Pre pochopenie si ešte ukážeme časť kódu \code{function begin(AeroShieldObject)}, v ktorej prebieha tvorba objektov pre arduino a hall senzor. Proces odvolávania sa na objekt \verb|arduino|, funguje cez názov triedy \verb|AeroShield|, v ktorom sa odvolávame na \verb|Object|. Príkaz \verb|arduino()| slúži na vyhľadanie dosky arduino, pripojenej do počítača a nastavenie parametrov potrebných na komunikáciu. Rovnako sa odvolávame aj na objekt \verb|as5600|, ktorý vytvoríme pomocou príkazu \verb|device()|, v ktorom zadávame objekt dosky arduino, a adresu zariadenia I$^2$C. Poslednou funkciou je vypísanie správy a konfiguráciu pinu \verb|AERO_UPIN| ako výstup. 

\begin{lstlisting}[caption={Knižnica AeroShield.m properties.},captionpos=b]
function begin(AeroShieldObject)          % funkcia inicializacie
	AeroShieldObject.arduino = arduino(); % tvorba arduino objektu
	AeroShieldObject.as5600 = device(AeroShieldObject.arduino,'I2CAddress',0x36); % tvorba objektu sonzoru
	configurePin(AeroShieldObject.arduino,AeroShieldObject.AERO_UPIN, 'DigitalOutput') % konfiguracia pinu ako vystup
	disp('AeroShield initialized.') % vypis spravu
end
\end{lstlisting}

Zostávajúce funkcie, ako aj celý zdrojový kód \verb|AeroShield.m| sa nachádza v prílohe na strane \pageref{AeroShield.m}. 

\newpage
\subsection{Simulink}

Simulink, narozdiel od MATLABU alebo Arduino IDE, využíva grafické programátorské prostredie, ktoré je založené na prepájaní funkčných blokov, vyberaných z prostredia knižnice. Pre prácu s Arduinom, v rámci Simulinku, existuje knižnica \textit{Simulink Support Package for Arduino Hardware}, ako aj knižnica AeroLibrary obr.\ref{OBRAZOK 2.6.4}, ktorú využívame priamo pre funkcie AeroShieldu. V knižnici AeroLibrary sa nachádzajú tieto bloky: 
\begin{itemize}
	\item Reference read- čítanie referenčnej hodnoty, 
	\item Actuator write- ovládanie akčného členu, 
	\item Angle read- meranie výstupu, 
	\item AeroShield- súhrn blokov na zapisovanie akčného zásahu ako aj meranie regulovanej veličiny.
\end{itemize}
Jednotlivé bloky majú svoju vlastnú vnútornú štruktúru a nastavenia, ktoré sa dajú meniť priamo v bloku, alebo pomocou masky bloku. Maska je akési grafické okno, ktoré sa zobrazí po kliknutí na blok. Toto okno zobrazuje informácie o funkcionalite bloku, ako aj prvky, pomocou ktorých sa dá nastaviť vnútorná štruktúra bloku. Pre lepšiu orientáciu medzi blokmi, má každý priradený vlastný ilustračný obrázok. Bližšie si o tvorbe masky povieme v časti \ref{Maska}. 

\begin{figure}[!tbh]
	\centering
	\includegraphics[width=125mm]{obr/AeroLib.png}
	\caption{Knižnica AeroLibrary.}\label{OBRAZOK 2.6.4}
\end{figure}

Blok \verb|Actuator Write| má najjednoduchšiu vnútornú štruktúru. Skladá sa z funkcie \verb*|Constrain|, ktorá prepočíta percentuálnu hodnotu vstupu na výstupný 8-bitový PWM signál. Tento signál posiela na Shield blok \verb*|PWM|, z knižnice \textit{Simulink Support Package for Arduino Hardware}. 

\subsubsection{Reference read}
\label{AngleRead}

V časti \verb|Reference read| máme na výber možnosť manuálnej, alebo automatickej trajektórie. Používateľ si z týchto možností vyberie pomocou tlačidiel v maske bloku. Vnútorná štruktúra sa ďalej skladá z troch podsystémov, z ktorých dva pre manuálnu trajektóriu obr.\ref{OBRAZOK 2.6.5} sú takmer totožné. Rozdiel medzi nimi je taký, že pri použití niektorých druhov Arduina\footnote[11]{Due, MKR1000, MKR1010, MKRZero, Nano 33IoT.} je pri použitý bloku \verb|Analog input|, výstup v tvare 12-bitového čísla, narozdiel od 10-bitového, ako je tomu pri ostatných podporovaných modeloch Arduina. Načítaná hodnota je ďalej upravovaná do percentuálnej podoby pomocou pre násobenia konštantou v bloku \verb*|gain|. 

Podsystém pre automatickú trajektóriu obr.\ref{} má v sebe prednastavené pole referenčných hodnôt. 

\begin{figure}[!tbh]
	\hfill
	\subfigure[Tri podsystémy bloku Reference read.]{\includegraphics[width=6cm]{obr/Subsystem.png}}
	\hfill
	\subfigure[Podsystém pre 10-bitovú logiku.]{\includegraphics[width=9cm]{obr/10bit.png}}
	\hfill
	\caption{Reference read- Simulink.}\label{OBRAZOK 2.6.5}
\end{figure}

\subsubsection{Angle read}

Tento blok obr.\ref{OBRAZOK 2.6.6} slúži na čítanie uhlu kyvadla, ako aj pre bezpečnostnú kontrolu uhlu kyvadla. Jeho vnútorná štruktúra sa dá rozdeliť na štyri menšie celky. 
\begin{itemize}
	\item Zapisovanie príkazov na senzor,  
	\item čítanie uhlu kyvadla zo senzoru, 
	\item kalibrácia nulového uhlu kyvadla,
	\item kontrola uhlu kyvadla. 
\end{itemize}

\begin{figure}[!tbh]
	\centering
	\includegraphics[width=\textwidth]{obr/AngleRead.png}
	\caption{Angle read- Simulink.}\label{OBRAZOK 2.6.6}
\end{figure}


\paragraph{I2C zápis}

Blok I2C Write obr.\ref{OBRAZOK 2.6.6}(ľavý dolný roh), z knižnice \textit{Simulink Support Package for Arduino Hardware}, zapisuje na senzor striedavo hodnoty 12 a 13, ktoré slúžia ako príkaz pre čítanie a následné odoslanie hodnoty pootočenia kyvadla. Prepínanie medzi týmito dvomi hodnotami zabezpečuje automatický prepínač (Automatic switch). 

\paragraph{I2C čítanie}

Samotné čítanie uhlu zabezpečuje blok I2C Read obr.\ref{OBRAZOK 2.6.6}(pravý horný roh), ktorého výstupom je 12-bitové číslo, predstavujúce aktuálne natočenie kyvadla. Tento výstup musíme previesť na uhlovú výchylku v rozmedzí od 0 do 360°. Tento prevod sa vykonáva v mapovacom podsystéme obr.\ref{OBRAZOK 2.6.7}. Mapovanie funguje na rovnakom princípe ako v príklade \ref{MatlabPID}. Jednotlivé premenné vchádzajú do bloku \verb*|MUX|, ktorý slúži ako multiplexer\footnote[12]{Zariadenie, ktoré vyberá medzi viacerými analógovými alebo digitálnymi vstupnými signálmi a tieto preposiela na jednu výstupnú linku.}. Signál z multiplexoru vstupuje do \verb|fcn bloku|, ktorý prepája Simulink s MATLABOM. Zdrojový kód \ref{MAPfcn} zobrazuje funkciu vo vnútri tohoto bloku. 


\begin{figure}[!tbh]
	\centering
	\includegraphics[width=\textwidth]{obr/SensMap.png}
	\caption{Mapovanie uhlu kyvadla- Simulink.}\label{OBRAZOK 2.6.7}
\end{figure}


\begin{lstlisting}[caption={Mapovacia funkcia vo fcn bloku.},captionpos=b,label=MAPfcn]
	function y = fcn(u)
	angMin=0;
	angMax=72;
	y = ((u(1) - u(2)) * (angMax - angMin)) / (u(3) - u(2)) + angMin ;
\end{lstlisting}


Proces získania premennej \verb|StartAngle|, ktorú využívame pri mapovaní, je opísaný v nasledujúcom odstavci. 

\paragraph{Kalibrácia} 
Kalibrácia prebieha vždy len raz a to pri spustení simulácie. V podsystéme (Enabled Subsystem) obr.\ref{OBRAZOK 2.6.8}.a, ktorý je spustený len ak je do neho privedený signál, sa nachádzajú funkcie pre zapisovanie a čítanie informácii zo senzoru. Uhol ktorý je týmto podsystémom zaznamenaný pri spustený simulácie, sa uloží v bloku \verb|Data store Write|, ako premenná \verb|StartAngle| obr.\ref{OBRAZOK 2.6.8}.b.                                                                                                                                                                                                                                                                                                                                                                                                                                                                                                                                                                                                                                                                                                                                                                                                                                                                                                                                                                                                                                                                                                                                                                                                                                                                                                                                                                                                                                                                                                                                                                                                                                                                                                                                                                                                                                                                                                                                                                                                                                                                                                                                                                                                                                                                                                                                                                                                                                                                                                                                                                                                                                                                                                                                                                                                                                                                                                                                                                                                                                                                                                                                                                                                                                                                                                                                                                                                                                                                                                                                                                                                                                                                                                                                                                                                                                                                                                                                                                                                                                                                                                                                                                                                                                                                                                                                                                                                                                                                                                                                                                                                                                                                                                                                                                                                                                                                                                                                                                                                                                                                                                                                                                                                                                                                                                                                                                                                                                                                                                                                                                                                                                                                                                                                                                                                                                                                                                                                                                                                                                                                                                                                                                                                                                                                                                                                                                                                                                                                                                                                                                                                                                                                                                                                                                                                                                                                                                                                                                                                                                                                                                                                                                                                                                                                                                                                                                                                                                                                                                                                                                                                                                                                                                                                                                                                                                                                                                                                                                                                                                                                                                                                                                                                                                                                                                                                                                                                                                                                                                                                                                                                                                                                                                                                                                                                                                                               

\begin{figure}[!tbh]
	\hfill
	\subfigure[Sústava pre zabezpečenie kalibrácie.]{\includegraphics[width=6cm]{obr/Calibrat.png}}
	\hfill
	\subfigure[Podsystém pre zaznamenanie nulového uhla.]{\includegraphics[width=9cm]{obr/calibDnu.png}}
	\hfill
	\caption{Kalibrácia- Simulink.}\label{OBRAZOK 2.6.8}
\end{figure}

\paragraph{Kontrola uhlu}

Kontrola uhlu kyvadla prebieha v podsystéme, ktorý je spustený päť sekúnd po začatí simulácie obr.\ref{OBRAZOK 2.6.6}(pravý dolný roh). Časový posun je použitý z dôvodu občasného šumu premenných počas začiatku simulácie. Tento šum môže spôsobiť nechcené výkyvy hodnoty uhlu kyvadla, ktoré by splnili podmienku v podsystéme a tým by zastavili priebeh simulácie. V podsystéme na obr.\ref{OBRAZOK 2.6.9}, sa nachádza blok \verb|Compare To Constant|, ktorý porovnáva hodnotu uhlu kyvadla, s daným maximálnym uhlom kyvadla. Táto hodnota je v našom prípade 110°. Pokiaľ kyvadlo dosiahne väčší uhol, ako je uhol dovolený, blok \verb|Stop Simulation| zastaví simuláciu.  

\begin{figure}[!tbh]
	\centering
	\includegraphics[width=100mm]{obr/AngleControl.png}
	\caption{Podsystém na kontrolu uhlu kyvadla.}\label{OBRAZOK 2.6.9}
\end{figure}

\newpage
\subsubsection{Tvorba masky}
\label{Maska}

Maska slúži ako vlastné používateľské rozhranie bloku resp. podsystému. Maskovaním bloku zapúzdrime blokovú schému tak, aby mala podla potreby, vlastné dialógové okná s nastaviteľnými parametrami, popisy bloku, výzvy na zadanie parametrov alebo texty s nápovedami. Vďaka maske blokov, sa tieto viacej podobajú na bloky, ktoré nájdeme v predprogramovaných knižniciach Simulinku. Masku upravujeme pomocou editora, ktorý spustíme pravým kliknutím na blok, s následným výberom možnosti \verb|Mask|. V maske sa nachádzajú štyri hlavné editovacie rozhrania. 

Prvým z nich je záložka \verb|Icon & Ports|. V tomto okne upravujeme hlavne vzhľad bloku, teda jeho zobrazenie, rotáciu, inicializáciu alebo aj ikona bloku. Pre všetky bloky v knižnici AeroLibrary, boli vytvorené vlastné ikony, ktoré sa do masky vkladajú nasledujúcim príkazom \code{image('assets/Pote.png')}.

Ďalšou v poradí je záložka \verb|Parameters & Dialog|. V tejto sekcii nastavujeme vzhľad a funkcie masky, ktorá sa zobrazí po kliknutí na blok. V prípade bloku \verb|Reference read|, máme v maske na výber voľbu automatickej alebo manuálnej referenčnej trajektórie. Medzi týmito možnosťami si vyberáme pomocou tlačidiel (Radio button) obr.\ref{OBRAZOK 2.6.100}, ktorých výber taktiež formátuje zobrazenie masky bloku. 

Pokiaľ si z tlačidiel vyberieme manuálnu trajektóriu, zobrazí sa ďalšia možnosť výberu, a tou je model dosky Arduino. Jedná sa o roletové, alebo "popup" menu, v ktorom si vyberieme typ dosky s ktorou používame AeroShield. Dôvod tohoto výberu je opísaný v časti \ref{AngleRead}, na strane \pageref{AngleRead}. Pokiaľ si zvolíme Automatickú referenčnú trajektóriu, možnosť voľby dosky sa nezobrazí. Táto funkcia je vykonávaná automaticky, zdrojovým kódom \ref{Callback}, ktorý sa nachádza v časti \verb|Callback|, v nastaveniach tlačidiel voľby trajektórie. 

\begin{lstlisting}[caption={Callback funkcia.},captionpos=b,label=Callback]
	ref = get_param(gcb,'ref');
	if strcmp(ref(1),'M')
	set_param(gcb,'MaskVisibilities',{'on';'on'}),
	else
	set_param(gcb,'MaskVisibilities',{'on';'off'}),
	end
\end{lstlisting}

Pomocou príkazu \verb|get_param|, sa nám do premennej \verb|ref|, uloží výber tlačidla v maske. Príkaz \verb|gcb| vráti cestu
daného bloku v aktuálnom systéme. Podmienka \verb|if| kontroluje ktorú z možností sme si vybrali, pomocou príkazu \verb|strcmp|, ktorý porovnáva prvé písmeno z premennej \verb|ref|, s písmenom 'M' (Manual). Pokiaľ sa tieto zhodujú, príkaz \verb|set_param|, spolu s príkazom \verb|MaskVisibilities|, nastaví obe polia tlačidiel ako viditeľné, teda s hodnotou \verb|'on'|. Pokiaľ sa písmená nezhodujú a teda bola vybraná trajektória automatická, možnosť výberu dosky Arduino, je skrytá príkazom \verb|'off'|. 


\begin{figure}[!tbh]
	\hfill
	\subfigure[Nastavenia masky.]{\includegraphics[width=10cm]{obr/ParamAnd.png}}
	\hfill
	\subfigure[Zobrazenie masky.]{\includegraphics[width=5.5cm]{obr/Maska.png}}
	\hfill
	\caption{Reference read- Simulink.}\label{OBRAZOK 2.6.100}
\end{figure}

\verb|Initialization| tvorí tretiu záložku v maske. V tejto záložke môžeme pridávať MATLAB kód, ktorý ovláda a formátuje podsystémy. V maske bloku \verb|Reference read |, máme v tejto záložke vložený kód \ref{Initialization}. Podľa pravdivosti \verb|if| podmienky, príkaz \newline \verb|set_param(gcb,'LabelModeActiveChoice'| aktivuje podsystém bloku, ktorého názov je uvedený ako tretí parameter v zátvorke príkazu. 

\begin{lstlisting}[caption={Initialization- maska Reference read.},captionpos=b,label=Initialization]
	ref = get_param(gcb,'ref');
	model = get_param(gcb,'model');
	if strcmp(ref,'Manual')
	if strcmp(model,'Due')
	set_param(gcb,'LabelModeActiveChoice', '12bit');
	elseif strcmp(model,'MKR1000')
	set_param(gcb,'LabelModeActiveChoice', '12bit');
	elseif strcmp(model,'MKR1010')
	set_param(gcb,'LabelModeActiveChoice', '12bit');
	elseif strcmp(model,'MKRZero')
	set_param(gcb,'LabelModeActiveChoice', '12bit');
	elseif strcmp(model,'Nano 33IoT')
	set_param(gcb,'LabelModeActiveChoice', '12bit');
	elseif strcmp(model,'Others') 
	set_param(gcb,'LabelModeActiveChoice', '10bit');
	end
	end
	if strcmp(ref,'Automatic')
	set_param(gcb,'LabelModeActiveChoice', 'Auto');
	end
\end{lstlisting}

Pre lepšie porozumenie blokov a ich funkcií, bol v API Simulink zostavený inštruktážny príklad \verb|AeroShieldOpenLoop| obr.\ref{OBRAZOK 2.6.111}. Tento príklad funguje na princípe merania hodnoty bežca potenciometra na Shielde, a následného zapisovania nameranej hodnoty na akčný člen sústavy. Zároveň je meraný uhol v ktorom sa kyvadlo nachádza. Obe tieto hodnoty sú priebežne zobrazované na grafe. 

\begin{figure}[!tbh]
	\centering
	\includegraphics[width=100mm]{obr/AeroOpenLoop.png}
	\caption{AeroShield\_OpenLoop.}\label{OBRAZOK 2.6.111}
\end{figure}

