\LARGE\bf{Zdrojový kód súboru AeroShieldPID.ino}
\label{AeroShieldPID.ino}
\vspace{1cm}
\begin{lstlisting}[numbers=left,basicstyle=\scriptsize,caption={Zdrojový kód súboru AeroShieldPID.ino.},captionpos=b]	
#include "AeroShield.h"              
#include <Sampling.h>   

#define MANUAL 0    
#define KP 1.7  
#define TI 3.8  
#define TD 0.25   

float startangle=0; 
float lastangle=0; 
float pendulumAngle;

unsigned long Ts = 3; 
unsigned long k = 0; 
bool nextStep = false;  
bool realTimeViolation = false;

int i=i;          
int T=1000;           
float R[]={45.0,23.0,75.0,32.0,58.0,10.0,35.0,19.0,9.0,43.0,23.0,65.0,15.0,80.0}; 
float r=0.0;          
float y = 0.0;        
float u = 0.0;         

void setup() {           
	Serial.begin(250000);                         
	AeroShield.begin(AeroShield.detectMagnet());
	startangle = AeroShield.calibration(AeroShield.getRawAngle()); 
	lastangle=startangle+1024;                                  
	Sampling.period(Ts*1000);      
	PIDAbs.setKp(KP);       
	PIDAbs.setTi(TI);    
	PIDAbs.setTd(TD);     
	PIDAbs.setTs(Sampling.samplingPeriod); 
	Sampling.interrupt(stepEnable); 
}

void loop() {
	if(pendulumAngle>120){
		AeroShield.actuatorWrite(0);
		while(1);
	} 
	if (nextStep) {    
		step();          
		nextStep = false;  
	}
}

void stepEnable() {             
	if(nextStep == true) {         
		realTimeViolation = true;   
		Serial.println("Real-time samples violated."); 
		analogWrite(5,0);  
		while(1);    
	}
	nextStep = true; 
}

void step() {  
	#if MANUAL                       
	r = AeroShield.referenceRead(); 
	#else        
	if(i>(sizeof(R)/sizeof(R[0]))) {  
		analogWrite(5,0); 
		while(1); 
	} else if (k % (T*i) == 0) {
		r = R[i];
		i++; 
	}
	#endif
	y= AutomationShield.mapFloat(AeroShield.getRawAngle(),startangle,lastangle,0.00,100.00);
	u = PIDAbs.compute(r-y,0,100,0,100);
	AeroShield.actuatorWrite(u);
	
	Serial.print(r);
	Serial.print(", ");
	Serial.print(y); 
	Serial.print(", ");
	Serial.println(u); 
	k++; 
}
\end{lstlisting}